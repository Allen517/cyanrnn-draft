\section{Introduction}

The emergence of Social Media platform has revolutionized dissemination of
information via its great ease in inforamtion delivery, accessing and filtering. 
In Social Media, pieces of information, posted by users spontaneously, propagate
along social relationships between users, explict or implict, forming cascade dynamics.
Modeling cascade dynamics is the fundamental to understand information
propagation and launch series of social applictions, i.e., viral marketing, popularity
prediction and rumor detection. 

The key to cascade dynamics modeling is to find a well-defined function in
hypothesis space based on observed cascades. Existed methods for this problem
fall into three main paradigms: pairwise, nodewise and eventwise modeling. The
majority works in cascade dynamics modeling focus on pairwise modeling,
defining the propagation probability of information between all pairs of
users~\cite{SaitoPredDiffusionProb08,goyal2010learning,gomez2013modeling}.
However, pairwise models suffer severe overfitting and overrepresentation
problems especially in sparse social data, proved in \cite{WangAAAI15}.
Nodewise modeling learn latent user-specific characteristics instead of pairwise
manners, effectively combating overfitting and overrepresentation problems.
Bourigault et.al.~\cite{bourigault2016representation} learn user-specific
latent space in Independent Cascade (IC) model. Wang et.al~\cite{WangAAAI15}
capture users' influence and susceptibility in latent space and define
propagation probability according to users latent characteristics.
Kurashima et.al.~\cite{KurashimaKDD14} embeded users into low-dimensional
visualization space in Continous Time Independent Cascade (CTIC) model. But
nodewise models require strong prior knowledge on generation processes of
cascades in order to better fit the observations. Recently, eventwise models
received great success in modeling sequence data. 

Eventwise methods aims to learn history embedding in order to model the
generation of next event, e.g., cascade. 
Manavoglu~\cite{Manavoglu2003userbehaviormodels} propose users behavior
generation method based on maxent and Markov mixture model. Recently, the
efficient way of eventwise modeling can be achieved by Recurrent Neural Network
(RNN)~\cite{bengio2003neural,goldberg2014word2vec,mikolov2010recurrent,sundermeyer2012lstm}.
Du et.al.~\cite{DuKDD2016} proposed a Recurrent Marked Temporal Point Process
(RMTPP) for event streams. The outputs of hidden layer in Recurrent Neural
Network (RNN) represents the embedding of the event histories,
then parameterizing the random process. The benefits of eventwise modeling are
two folds: 1) avoiding strong prior knowledge on models and networks with
respect to different observed cascades; 2) enlarging the functional space when
searching the optimal cascade dynamics models, which may have great probability
to better model cascade dynamics. 

Despite of advantages in eventwise modeling, the traditional sequence models
may meet ``crossing dependency'' problem in cascade dynamics modeling.
The crossing dependencies problem is mainly caused by tree structure of
propagation. Fig.~\ref{fig:mot} shows two typical crossing dependency
cases in practical. For modeling dependence between 1st and 3rd event, we must
use redundant information passing from 2nd event, called
``redundant modeling''. If we abandon useful information inherited by 3rd
event when modeling the 4th event, the generation of 5th event would lose
useful information from 3rd event, called ``cut-off modeling''. Crossing
dependency problems limit the efficiency of sequence modeling.

\begin{figure}[ht!]
\label{fig:mot}
\centering
\includegraphics[width=0.45\textwidth]{figs/mot.eps}
\caption{Tree structure of propagation and crossing dependency problems in
sequence modeling. For modeling dependence between 1st and 3rd event, we must
use redundant information passing from 2nd event, called
``redundant modeling''. If we abandon useful information inherited by 3rd
event when modeling the 4th event, the generation of 5th event would lose
useful information from 3rd event, called ``cut-off modeling''.
}
\end{figure}

In this paper, we propose a \textbf{C}ascade d\textbf{Y}namics modeling with
\textbf{A}ttentio\textbf{N}-based RNN, named (CYAN-RNN). We construct a pooling
layer above the output of hidden layer in RNN, aggregating event embedding in
history. The weights in pooling layer pointing to each historical event
embedding refers to connections between current event and history. We choose
attention mechanism~\cite{bahdanau2014neural} to realize the pooling layer,
automatically learned the connection weights. The benifits of our proposed model
are three-fold: 1) We propose a eventwise method, using sequence modeling, for
cascade dynamics modeling; 2) We point out crossing dependency problem in
traditional sequence modeling when model cascade dynamics. Thus, we
proposed CYAN-RNN to solve the problem; 3) We conduct experiments on synthetic
and real-world datasets to show that our model consistently outperform than
previous modeling methods in cascade dynamics modeling.

% to limit hypothesis space 
% 
% require strong prior knowledge on generation of cascade data in
% order to determine specific form of cascade model as the limitation of
% hypothesis space.

% To overcome the drawbacks in pairwise modeling, n
% 
% Nodewise modeling and
% eventwise modeling.
% 
% Existed methods for this problem
% fall into three main paradigms. 

% In Social Media, users can post  information spontaneously, spreading 
% 
% The emergence of Social Media platform has motivated a large mount of
% hot researches, including Social Network Analysis, Marketing and Information
% Propagation. 
