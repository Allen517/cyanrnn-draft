\section{Introduction}

The emergence of Social Media platform has revolutionized dissemination of
information via its great ease in inforamtion delivery, accessing and filtering. 
In Social Media, pieces of information, posted by users spontaneously, propagate
along social relationships between users, explict or implict, forming cascade dynamics.
Modeling cascade dynamics is the fundamental to understand information
propagation and launch series of social applications, i.e., viral marketing,
popularity prediction and rumor detection. 

In order to model the cascades, many information diffusion models have been
proposed in the literature, such as discrete-time independent cascade model,
discrete-time linear threshold model and continuous-time independent cascade
model. A diffusion model is often associated with a directed graph and 
parameterized by propagation probabilities defined on edges or interpersonal
influence defined on nodes.
% interpersonal influence defined on edges or
% interactions between users. 
Kempe et al.~\cite{kempe2003maximizing} implemented
the linear threshold model by a degree-modulated edge weight. Goyal et
al.~\cite{goyal2010learning} provided two static models in terms of Bernoulli
distribution and Jaccard index, and learned temporal factors to maximize
likelihood of cascades. Gomez-Rodriguez et al.~\cite{gomez2013modeling}
estimated a transimission function over time associated with each edge, assuming
that observed diffusion time is independently sampled according to the
transimission function. Wang et.al~\cite{WangAAAI15}
captured users' influence and susceptibility in latent space and modeled the
transition probabilities of users' activated status according to users' latent
characteristics.

% survival model 
% and a
% cascade from a model is just a set of influenced nodes according to the model
% given a set of source nodes. 
% 
% The majority of works in cascade dynamics modeling is based on information
% diffusion models.
% 
% A diffusion model is often associated with a directed graph and a cascade from a
% model is just a set of influenced nodes according to the model given a set of
% source nodes. In gerneral, we have the following three typical types of
% diffusion models: discrete-time independent cascade model, discrete-time linear threshold
% model, continuous-time independent cascade model and 
% 
% The key to cascade dynamics modeling is to find a well-defined 
% function in hypothesis space based on observed cascades and forecast following
% propagations. Existed methods for this problem fall into two main paradigms:
% interpersonal influence modeling and sequential modeling. The majority of works
% focus on interpersonal influence modeling,  defining activation
% function on users, 

% The key to cascade dynamics modeling is to find a well-defined function in
% hypothesis space based on observed cascades. Existed methods for this problem
% fall into three main paradigms: pairwise, nodewise and eventwise modeling. The
% majority works in cascade dynamics modeling focus on pairwise modeling,
% defining the propagation probability of information between all pairs of
% users~\cite{SaitoPredDiffusionProb08,goyal2010learning,gomez2013modeling}.
% However, pairwise models suffer severe overfitting and overrepresentation
% problems especially in sparse social data, proved in \cite{WangAAAI15}.
% Nodewise modeling learn latent user-specific characteristics instead of pairwise
% manners, effectively combating overfitting and overrepresentation problems.
% Bourigault et.al.~\cite{bourigault2016representation} learn user-specific
% latent space in Independent Cascade (IC) model. Wang et.al~\cite{WangAAAI15}
% capture users' influence and susceptibility in latent space and define
% propagation probability according to users latent characteristics.
% Kurashima et.al.~\cite{KurashimaKDD14} embeded users into low-dimensional
% visualization space in Continous Time Independent Cascade (CTIC) model. But
% nodewise models require strong prior knowledge on generation processes of
% cascades in order to better fit the observations. Recently, eventwise models
% received great success in modeling sequence data. 

\begin{figure}[t]
\label{fig:mot}
\centering
\includegraphics[width=0.45\textwidth]{figs/mot.eps}
\caption{
An example of crossing dependency problem in sequence modeling. 
% Tree structure of propagation and crossing dependency problems in
% sequence modeling. For modeling dependence between 1st and 3rd event, we must
% use redundant information passing from 2nd event, called
% ``redundant modeling''. If we abandon useful information inherited by 3rd
% event when modeling the 4th event, the generation of 5th event would lose
% useful information from 3rd event, called ``cut-off modeling''.
}
\end{figure}

However, the information diffusion is too complicated to easily determine
the most appropriate diffusion model in practice. The misspecified diffusion
model would lead to large model bias. Besides, the diffusion network structure
can be also hidden from us. We need to learn the underlying
diffusion network structure with respect to parameter estimation in diffusion
model at most cases. It often leads to under-determined high computational cost
where specialized methods need to be designed. 

% Real world information diffusion is complicated, and it is not easy to determine
% the most suitable diffusion model in practice. A chosen diffusion model may be
% misspecified compared to real world data and lead to large model bias. 
% 
% The diffusion network structure can be also hidden from us, so we need to learn
% not only the paramters in the diffusion model, but also the diffusion network
% structure. This often leads to under-determined high dimensional estimation
% problem where specialized methods need to be designed. 

Instead of learning a particular diffusion model, sequence modeling can be
used to discover complex dependencies and patterns between the following
propagation and current histories, thus modeling the cascade dyanmics.
For example, Manavoglue~\cite{Manavoglu2003userbehaviormodels} proposed a user
behavior generation method based on maxent and Markov mixture model. Recently,
the efficient way of sequence modeling can be achieved by Recurrent Neural
Network
(RNN)~\cite{goldberg2014word2vec,mikolov2010recurrent,sundermeyer2012lstm}.
Du et al.~\cite{DuKDD2016} proposed a RNN framework, called RMTPP,
where the embedding of hidden units are used to parameterize the generation
process of propagations. The benefits of sequence modeling are two-fold: 1)
It avoids strong prior knowledge on diffusion model; 2) It has no limitation on
hidden social network structure. 

% The benefits of eventwise modeling are
% two folds: 1) avoiding strong prior knowledge on models and networks with
% respect to different observed cascades; 2) enlarging the functional space when
% searching the optimal cascade dynamics models, which may have great probability
% to better model cascade dynamics. 
% 
% Eventwise methods aims to
% learn history embedding in order to model the generation of next event, e.g., cascade. 
% Manavoglu~\cite{Manavoglu2003userbehaviormodels} propose users behavior
% generation method based on maxent and Markov mixture model. Recently, the
% efficient way of eventwise modeling can be achieved by Recurrent Neural Network
% (RNN)~\cite{bengio2003neural,goldberg2014word2vec,mikolov2010recurrent,sundermeyer2012lstm}.
% Du et.al.~\cite{DuKDD2016} proposed a Recurrent Marked Temporal Point Process
% (RMTPP) for event streams. The outputs of hidden layer in Recurrent Neural
% Network (RNN) represents the embedding of the event histories,
% then parameterizing the random process. The benefits of eventwise modeling are
% two folds: 1) avoiding strong prior knowledge on models and networks with
% respect to different observed cascades; 2) enlarging the functional space when
% searching the optimal cascade dynamics models, which may have great probability
% to better model cascade dynamics. 

Despite of advantages in sequence modeling, the traditional methods
may meet ``crossing dependency'' problem in cascade dynamics modeling, mainly
caused by
%  The crossing dependencies problem is mainly caused by 
tree structure of propagation. Fig.~\ref{fig:mot} shows the crossing
dependency dilemma in practical. 
As consideration
of tree strucutre of propagations, we need to construct the dependence between
the 1st and the 3rd propagations, causing redundant information usage passing
from the 2nd one (the transitive dependence marked as grey in time
line). If we abandon information inherited from the 2nd propagation, the
generation of 4th propagation will lose useful information from the 2nd
one as the corresponding user $u_4$ is directly connected to $u_2$ on the
cascade tree
(the transitive dependence marked as red in time line). 
% For modeling the dependence between the 1st
% and 3rd propagation, we must use redundant information passing from the 2nd
% one.
% % , called ``redundant modeling''. 
% Otherwise, if we abandon useful information inherited from the 2nd propagatoin,
% the generation of the 4th propagation will lose some useful information as the
% corresponding user $u_4$ is directly connected to $u_2$ on the cascade tree.
% event when modeling the 4th event, the generation of 5th event would lose
% useful information from 3rd event, called ``cut-off modeling''. Crossing
% dependency problems limit the efficiency of sequence modeling.
In this paper, we propose a \textbf{C}ascade d\textbf{Y}namics
modeling with \textbf{A}ttentio\textbf{N}-based RNN (CYAN-RNN) to better
apply RNN in cascade dynamics modeling. We construct a dynamic pooling
layer above hidden units of RNN, aggregating all embeddings of hidden units
until current step. The generation of following propagation is relied on the
aggregated embedding. 
% The weights in pooling layer pointing to each historical event
% embedding refers to connections between current event and history. 
We choose attention mechanism~\cite{bahdanau2014neural} to realize the pooling layer,
where the parameters can be automatically learned by maximizing the
likelihood of cascade. Moreover, we propose coverage in attention mechanism to
ensure sufficient usage of each historical embedding involved in generation of
following propagations.
% CYAN-RNN(cov) to construct coverage
% on attention mechanism in order to solve over-dependent and under-dependent
% problem existed in CYAN-RNN. 
% With consideration of preference 
% it is a common phenomenon that users may have a
% higher probability activated by recent propagation than past
% The benifits of our proposed model are three-fold: 
More specifically, our work makes the following contributions:
\begin{itemize}
  \item We exploit crossing dependency problem existed in traditional sequence
  modeling when applied to model cascade dynamics and proposed attention-based
  RNN to solve the problem;
  \item The alignments in proposed attention mechanism are homologous to the
  underlying structure of propagation.
  \item The proposed coverage in attention mechansim further improve the
  performance of attention mechanism, including cascade prediction and
  propagation structure inference.
%    and also  We propose coverage
%   in attention mechansim to further improve the performance of attention mechanism;
\end{itemize}
% 1) We exploit crossing dependency problem existed in traditional sequence
% modeling when applied to model cascade dynamics and proposed attention-based 
% RNN to solve the problem;
% We propose two sequence modeling methods based on RNN for cascade dyanmics
% modeling;
% eventwise method, using sequence
% modeling, for cascade dynamics modeling; 
% 2) We propose attention mechansim with coverage to further improve the
% performance of attention mechanism;
% 3) The alignments in proposed attention mechanism are homologous to the
% underlying structure of propagation.   
% 
% We exploit crossing dependency problem existed in traditional sequence
% modeling when applied to cascade dynamics modeling. The alignments in
% proposed attention mechanism can 
% 
% point out crossing dependency problem in
% traditional sequence modeling when model cascade dynamics. Thus, we
% proposed CYAN-RNN to solve the problem; 
We conduct extensive experiments on synthetic
and real-world datasets. Compared with several widely-used methods in cascade
dynamics modeling, our proposed models consistently outperform them at
cascade prediction.
% predicting next propagations. 

% we exploit quantitative analysis to
% understand 
% the experimental results show that our model consistently
% outperform than previous modeling methods in cascade dynamics modeling.

% to limit hypothesis space 
% 
% require strong prior knowledge on generation of cascade data in
% order to determine specific form of cascade model as the limitation of
% hypothesis space.

% To overcome the drawbacks in pairwise modeling, n
% 
% Nodewise modeling and
% eventwise modeling.
% 
% Existed methods for this problem
% fall into three main paradigms. 

% In Social Media, users can post  information spontaneously, spreading 
% 
% The emergence of Social Media platform has motivated a large mount of
% hot researches, including Social Network Analysis, Marketing and Information
% Propagation. 
