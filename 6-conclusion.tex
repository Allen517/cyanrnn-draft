\section{Conclusion}

In this paper, we present the cascade dynamics modeling with attention-based
RNN. As we know, it is a prior attempt on CDM based on RNN, which embed
historical propagations into vectors and then determine the next propagation sequentially. 
Instead of traditional interpersonal influence modeling, RNN can
be used to discover complex dependencies and patterns between the following
propagation and current histories. 
% Instead of traditional interpersonal influence modeling, sequential modeling can
% be used to discover dependencies and patterns between the following propagation and
% current histories. 
However, RNN suffers crossing dependency problem when applying
in CDM. To solve the problem, we propose attention mechanism
above hidden units of RNN, named CYAN-RNN, aggregating all current historical
embeddings. Thus, the generation of next propagation can directly depend on all
current histories instead of transitive dependent way. Moreover, we propose
CYAN-RNN(cov) to construct coverage on attention mechanism in order to solve
over-dependent and under-dependent problem existed in CYAN-RNN. In experiments,
we evaluate the effectiveness of our proposed model on both synthetic and real
datasets. Experimental results demonstrate that our proposed models can
consistently outperform compared modeling methods at both predicition tasks of
next activated user and time. Addtionallly, CYAN-RNN(cov) performs
better or even than CYAN-RNN on both synthetic and real datasets, proving that
coverage can help to efficiently utilize historical embeddings in attention
mechanism. Besides, we conduct experiments to exploit alignment quality. The
results show that the alignments from our proposed models can reflect
true propagation structures, which may be well applicable in practice.

% have some potential applications in
% practice.
% four synthetic datasets
% produced from different networks and propagation models and one real dataset. 
% crossing-dependency way. We propose CYAN-RNN 
% 
% Based on RNN, the proposed model can embed historical propagations
% into vectors and the next propagation 
